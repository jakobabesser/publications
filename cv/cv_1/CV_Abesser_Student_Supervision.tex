\documentclass[8pt,a4paper]{article}

%\usepackage[latin1]{inputenc}
\usepackage{german}
\usepackage{a4wide}
\usepackage{url}
\usepackage{graphicx}

\parindent 0cm
%\oddsidemargin 0.04cm
\topmargin -1.5cm
\oddsidemargin -1cm
\textwidth 17cm
\textheight 30cm

\newenvironment{itemizePacked}{
\begin{itemize}
  \setlength{\itemsep}{1pt}
  \setlength{\parskip}{3pt}
  \setlength{\parsep}{0pt}
  \renewcommand{\labelitemi}{$\bullet$}
}{\end{itemize}}

\newenvironment{itemizePacked2}{
\begin{itemize}
  \setlength{\itemsep}{1pt}
  \setlength{\parskip}{0pt}
  \setlength{\parsep}{0pt}
  \renewcommand{\labelitemi}{$-$}
}{\end{itemize}}

\begin{document}

\pagestyle{empty}


{\bf Supervised Students Works (Dr. Jakob Abe{\ss}er)}
\begin{itemizePacked}
\item Alexander Belz, Lorenz Hüttner, Gregor Siegert: Verbesserung bestehender Verfahren zur Instrumentenerkennung durch die Detektion monotimbraler Bereiche innerhalb von Musikst\"ucken (Media project, TU Ilmenau, 2009)
\item Michael Stein: Entwicklung eines Verfahrens zur Detektion und Neutralisation verschiedener Effekte auf Bass- und Gitarrenaufnahmen innerhalb von Musikst\"ucken (Diploma thesis, TU Ilmenau, 2009). 
\item Matthias Kahl: Zur Klassifikation von Hauptinstrumenten aus polyphonen und multitimbralen Musikaufnahmen (Diploma thesis, Fachhochschule Schmalkalden, 2009)
\item Anna Kruspe: Automatic Classifcation of Musical Pieces into Global Cultural Areas (Diploma thesis, TU Ilmenau, 2010)
\item Johannes Krasser: Automatische Detektion von Spielfehlern in Aufnahmen von polyphonen Instrumenten (Bachelor thesis, TU Ilmenau, 2010)
\item Thomas V\"olkel: Automatische Klassifikation lateinamerikanischer Musik durch charakteristische rhythmische Pattern und rhythmische High-Level-Features (Diploma thesis, TU Ilmenau, 2010)
\item Martin Herzog: Harmonieanalyse von MIDI-Daten als Grundlage f\"ur die Extraktion von harmonischen High-Level-Merkmalen (Diploma thesis, HU Berlin, 2011)
\item David Wagner: Implementierung und Evaluation einer interaktiven Fingersatz-Animation in Musiklernsoftware (Diploma thesis, TU Ilmenau, 2011)
\item Thomas Thron: Untersuchung und Entwicklung von Algorithmen zur tonalen und motivischen Segmentierung von Musikstücken (Bachelor thesis, FH Erfurt, 2011)
\item Patrick Kramer: Entwicklung eines Verfahrens zur automatischen Klangsynthese von Bassnoten unter Ber\"ucksichtigung typischer Spieltechniken des E-Basses (Diploma thesis, TU Ilmenau, 2011)
\item Carsten B\"onsel: Implementierung eines Videoanalyseverfahrens zur automatischen Erkennung der Position der Greifhand in Gitarrenaufnahmen (Bachelor thesis, TU Ilmenau, 2011)
\item Markus Schubert: Klangsynthese von Gitarrensignalen (Bachelor thesis, TU Ilmenau, 2011)
\item Vedant Dhandhania: Tracking Perceived Loudness and Brightness in Audio Signals (Bachelor Thesis, Manipal University, 2011)
\item Johannes Krasser: Implementierung und Untersuchung von Merkmalen und Algorithmen f\"ur die Berechnung musikalischer Ähnlichkeit auf Basis von Klangobjekten (Master thesis, TU Ilmenau, 2012)
\item Martin D\"orr: Verlaufsbezogene Stimmungsannotation von Musikst\"ucken (Diploma thesis, TU Ilmenau, 2012)
\item Mikus Grasis: Verbesserung bestehender Verfahren zur Instrumentenerkennung in polyphonen
Musikaufnahmen (Bachelor thesis, Hochschule Emden / Leer, 2013)
\item Christian Kehling: Entwicklung eines parametrischen Instrumentencoders basierend auf Analyse und Re-Synthese von Gitarrenaufnahmen (Diploma thesis, TU Ilmenau, 2013)
\item Arndt Eppler: Entwicklung eines Verfahrens zur Audiorestauration basierend auf Re-Synthese von Gitarrenaufnahmen (Bachelor thesis, TU Ilmenau, 2013)
\item Andreas M\"annchen: Entwicklung eines echtzeitf\"ahigen Verfahrens zur automatischen Saitenerkennung in monophonen und polyphonen Gitarrenaufnahmen (Bachelor thesis, TU Ilmenau, 2013)
\end{itemizePacked}
\pagebreak

{\bf Supervised Students Works  (Dr. Jakob Abe{\ss}er)}

\begin{itemizePacked}
\item Konstantin Brand: Entwicklung eines Verfahrens zur automatischen Transkription von Walking Bass - Linien aus kommerziellen Jazzaufnahmen (Bachelor thesis, TU Ilmenau, 2015)
\item Sebastian Preu{\ss}e: Entwicklung eines Algorithmus zur segmentweisen Klassifkation von Instrumentenfamilien (Bachelor thesis, TU Ilmenau, 2015)
\item Daniel Matz: Entwicklung eines Verfahrens zum Automatic Remixing alter Jazzaufnahmen (Bachelor thesis, Fachhochschule D\"usseldorf, 2015)
\item Carsten B\"onsel: Development and Implementation of a Method for Automatic Best-Take Detection in Monophonic Vocal and Guitar Recordings (Master thesis, TU Ilmenau, 2015)
\item Arndt Eppler: Entwicklung und Implementierung eines Verfahrens zur automatischen und echtzeitf\"ahigen Erkennung von wiederholten rhythmischen Patterns sowie der rhythmischen Stilistik von Gitarrensignalen (Master thesis, TU Ilmenau, 2015)
\item Andreas M\"annchen: Entwicklung und Implementierung eines Verfahrens zur automatischen und echtzeitf\"ahigen Erkennung von Akkorden sowie wiederholten Harmoniefolgen in Gitarrensignalen (Master thesis, TU Ilmenau, 2015)
\end{itemizePacked}
%\vspace{2cm}
%
%Bonn, Januar 2007,
\end{document}



